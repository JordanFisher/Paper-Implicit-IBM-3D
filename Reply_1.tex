\documentclass[12pt]{article}
\usepackage{graphicx}
\usepackage{amssymb}
\usepackage{epstopdf}
\DeclareGraphicsRule{.tif}{png}{.png}{`convert #1 `dirname #1`/`basename #1 .tif`.png}

\textwidth = 6.5 in
\textheight = 9 in
\oddsidemargin = 0.0 in
\evensidemargin = 0.0 in
\topmargin = 0.0 in
\headheight = 0.0 in
\headsep = 0.0 in
\parskip = 0.2in
\parindent = 0.0in

\newtheorem{theorem}{Theorem}
\newtheorem{corollary}[theorem]{Corollary}
\newtheorem{definition}{Definition}

\title{Reply to Reviewer \# 1}
\author{}
\begin{document}
\maketitle
We would like to thank you,  reviewer,  for your thoughtful  comments and suggestions that helped to improve  the quality of our manuscript. 
We have done a revision to our original work to address all the issues you brought up.


We list below how each of the specific issues were addressed in our revised version. Your questions and comments are included in italics as reference.



\begin{enumerate}
\item {\em How would the scheme perform for problems involving flexible fibers
(e.g., comprised of linear springs) or bending-resistant fibers (e.g.,
comprised of linear beams)?  Would the scheme remain competitive in such
cases?}

The difficulty of solving the implicit system depends on
the nature of the fiber force function. This was discussed in some detail in our 2D work [JCP, 228(19), 7137--7158, 2009]. In particular, 
if the Jacobian of the force density function is negative definite then  Newton's method, in concert with a fast methodology to evaluate
flow-structure interactions, works extremely well. 
Forces with a non-definite Jacobian are more difficult to handle. In our 2D work,  we developed a method to treat
the special case of non-definite forces from highly rigid fibers. While very effective in 2D, we have not yet
 extended this approach to 3D. 
 
 From our perspective, there are two interrelated problems in the construction of effective, non-stiff IB methods: 1.~To expedite flow-structure interactions and 2.~To solve the implicit system. The approaches discussed in  [JCP, 228(19), 7137--7158, 2009] to deal with Problem 2 apply, in principle, to both the 2D and the 3D cases. In view of this, 
 we have focused our attention on Problem 1 for the current work.  We have revised the Introduction as well as  Section 4. (Solution Methods) to better convey these points. 


\item {\em The method also crucially relies on the discrete Green's function
being well approximated as translationally invariant, and on the use of
interpolation of precomputed tabulation of Green's function values.
Although the authors demonstrate analytically that these approximations do
not degrade the order of accuracy of the method, they both seem to be of a
form that could exacerbate the sensitivity of the numerical results to the
relative position of the immersed structure with respect to the fluid grid
at practical grid spacings. It might be useful if the authors could
provide some numerical results addressing this issue.}

We have now added a subsection (6.3) to document numerically the overall accuracy of the proposed implicit methodology as compared with the explicit approach. 



\item {\em Finally, a number of minor questions: At what Reynolds numbers are the flows in the numerical examples?  How does the performance of the solver depend on the Reynolds number (if it does at all)?}

The simulations in the paper have $Re \approx 10$.  Based on the robustness of the semi-implicit discretization (see e.g. [8]), we expect the proposed methodology  to be robust with respect to $Re$. However, we have only tested it for moderate to small, including $Re \rightarrow 0$ (Stokes flow) Reynolds numbers and observed the same high efficiency. We have added a note in the revised manuscript to state this explicitly.
 

\item {\em How does the performance of the solver depend on the timestep size? 
Why use a fixed timestep size in the numerical results when presumably the
scheme is stable up to some CFL number that is approaching 1?}

 We chose $ \Delta t=O(h)$ to maintain overall first order accuracy. 
\end{enumerate}


 \end{document}
 
