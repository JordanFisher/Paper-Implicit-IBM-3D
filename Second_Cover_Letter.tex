\documentclass[12pt]{article}
\usepackage{graphicx}
\usepackage{amssymb}
\usepackage{epstopdf}
\DeclareGraphicsRule{.tif}{png}{.png}{`convert #1 `dirname #1`/`basename #1 .tif`.png}

\textwidth = 6.5 in
\textheight = 9 in
\oddsidemargin = 0.0 in
\evensidemargin = 0.0 in
\topmargin = 0.0 in
\headheight = 0.0 in
\headsep = 0.0 in
\parskip = 0.2in
\parindent = 0.0in

\newtheorem{theorem}{Theorem}
\newtheorem{corollary}[theorem]{Corollary}
\newtheorem{definition}{Definition}

\newcommand{\Comment}[1]{}
\newcommand{\floor}[1]{\left\lfloor #1 \right\rfloor}
\newcommand{\full}{\displaystyle}
\newcommand{\norm}[1]{\left\lVert#1\right\rVert}
\newcommand{\Inf}{\infty}
\newcommand{\B}[1]{\mathbf{#1}}
\newcommand{\C}[1]{\mathcal{#1}}
\newcommand{\BB}[1]{\mathbb{#1}}
\newcommand{\ti}[1]{\tilde{#1}}
\renewcommand{\to}{\rightarrow}
\newcommand{\LL}{\left\langle}
\newcommand{\RR}{\right\rangle}

\title{}
\author{}
\begin{document}
\maketitle
{\bf Dear Prof. Anna-Karin Tornberg,}

Dear Prof.Tornberg

We have now done a revision to address the minor points for needed further
clarification. 
We detail below how each point/question was addressed.
Many thanks for your excellent handling of out manuscript.

With best regards, \\

Jordan E. Fisher and Hector D. Ceniceros. \\ \\ \\


Editor's Question:

In your comment to me you acknowledge that ``For bounded domains, we will lose the (near)-translation invariance in Gh which is critical for our expedited
approximation.'' I can however not find this information in this manuscript? I think it should be pointed out clearly to the reader as well, and ask you to insert a comment in the introduction where you have said that you develop the method ``for periodic boundary conditions as it is customary in the IB setting'', and then more specifically return to the issue when the specifics are described in a later section.

Response:

We have revised both the introduction and the initial discussion of translation
invariance in Section 5 to mention the requirements for having approximate
translation invariance and the implications of those requirements. Specifically, in the introduction we add

``The assumption of periodic boundary conditions is critical to one aspect of the
proposed method (the near translation invariance). However, Dirichlet type of
boundary conditions can be easily implemented in the IB framework by treating solid boundaries as being immersed within a periodic domain.'' 

and in Section 5

``We note that translation invariance here is dependent on the type of boundary
conditions on our domain $\Omega$. In particular, $\Omega$ must be either periodic or infinite $(\Omega = \BB{R}^3)$. This requirement is not as restrictive as it might seem, as Dirichlet type of boundary conditions can be  easily implemented in the IB framework as immersed boundaries within a periodic domain.''

\end{document}
 
