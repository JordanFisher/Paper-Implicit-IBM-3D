\documentclass[12pt]{article}
\usepackage{graphicx}
\usepackage{amssymb}
\usepackage{epstopdf}
\DeclareGraphicsRule{.tif}{png}{.png}{`convert #1 `dirname #1`/`basename #1 .tif`.png}

\textwidth = 6.5 in
\textheight = 9 in
\oddsidemargin = 0.0 in
\evensidemargin = 0.0 in
\topmargin = 0.0 in
\headheight = 0.0 in
\headsep = 0.0 in
\parskip = 0.2in
\parindent = 0.0in

\newtheorem{theorem}{Theorem}
\newtheorem{corollary}[theorem]{Corollary}
\newtheorem{definition}{Definition}

\title{Reply to Reviewer \# 3}
\author{}
\begin{document}
\maketitle
We would like to thank you,  reviewer,  for your thoughtful  comments and suggestions that helped to improve  the quality of our manuscript. 
We have done a revision to our original work to address all the issues you brought up.


We list below how each of the specific issues were addressed in our revised version. Your questions and comments are included in italics as reference.



\begin{enumerate}
\item {\em Figure 3 concerns a steady test case. The figure plots the
computational error against the number of terms in the SVD expansion p.
How is the error computed? Is the exact solution known? The figure seems
to plot the absolute error; if so, it would be helpful to also give the
relative error, or at least the order of magnitude, e.g. is it larger or
smaller than 1\%? It seems the error is O(h), where h is the grid spacing;
is this correct? Can this be demonstrated numerically by plotting results
for several values of h?}

 The plot is comparing the treecode to a discrete fluid solve, not
to an exact solution to the continuous equations. The relative difference
is roughly 5\%, depending on $h$. This error is smaller than the error of our
discretizations. We have added an additional subsection (6.3) to provide more data about the treecode's accuracy. In
particular,  Figure 4 gives a numerical verification that the treecode
error is $O(h)$. A paragraph was also added to Section 5.3 to further explain this.



\item {\em Tables 2 and 5 concern unsteady test cases and show remarkable speedup
for the proposed method in comparison with a standard explicit method.
However there needs to be some discussion of the accuracy of these
results, beyond the fact that the structure remains intact.}

This is included in our new subsection (6.3).



\end{enumerate}


 \end{document}
 
